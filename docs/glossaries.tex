%!TEX root = ../thesis.tex

%------------------------------------------------------------------------------%
% Defining new acronyms
% \newacronym{IDENTIFIER}{ABBREVIATION}{DESCRIPTION}
\newacronym{dhbw}{DHBW}{Duale Hochschule Baden-Württemberg}

\newacronym{http}{HTTP}{Hyper Text Transfer Protocol}
\newacronym{uri}{URI}{Uniform Resource Identifier}
\newacronym{url}{URL}{Uniform Resource Locator}

\newacronym{rest}{REST}{Representational State Transfer}
\newacronym{html}{HTML}{Hyper Text Markup Language}
\newacronym{css}{CSS}{Cascading Style Sheets}

\newacronym{ui}{UI}{User Interface}

\newacronym{sql}{SQL}{Structured Query Language}

\newacronym{csv}{CSV}{Comma-separated values}
\newacronym{json}{JSON}{JavaScript Object Notation}
\newacronym{xml}{XML}{Extensible Markup Language}
\newacronym{yaml}{YAML}{Yet Another Markup Language}
%------------------------------------------------------------------------------%

%------------------------------------------------------------------------------%
% Defining new glossary entries
% \newglossaryentry{IDENTIFIER}{name=NAME, description=DESCRIPTION}
\newglossaryentry{mit-lizenz}{
    name=MIT-Lizenz,
    description={
        Die MIT-Lizenz ist eine permissive Open-Source-Lizenz, die es erlaubt,
        dieses Projekt zu verteilen, zu verändern und vieles mehr. Die volle
        Lizenz ist im Root-Verzeichnis dieses Projekts zu finden.
    }
}

\newglossaryentry{github}{
    name=GitHub,
    description={
        GitHub ist eine Plattform zur Versionsverwaltung von Softwareprojekten.
        Projekte sind dort in sogenannten \emph{Repositories} organisiert. Der
        Quellcode ist öffentlich. Eine entsprechende Lizenzierung ermöglicht
        die Nutzung dieser Softwareporjekte auch für eigene Projekte.
    }
}
%------------------------------------------------------------------------------%