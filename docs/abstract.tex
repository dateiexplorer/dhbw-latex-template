%!TEX root = ../thesis.tex

%------------------------------------------------------------------------------%
% Here you can define your abstract in various languages, following this
% examples.
% Make sure, that you included the `abstract` package in your document file.
%------------------------------------------------------------------------------%

\begin{otherlanguage}{ngerman}
	\begin{abstract}
		Dieses Dokument bietet einen Überblick über verschiedene Funktionen der
        Vorlage für Projektarbeiten an der \acs{dhbw}. Gleichzeitig wurde
        dieses Dokument mithilfe der Vorlage für Projektarbeiten erstellt und
        dient damit als kleine Dokumentation und als Nachschlagewerk für
        verschiedene \LaTeX{}-Kommandos, um einen schnellen Einstieg auch ohne
        Vorkenntnisse in \LaTeX{} zu ermöglichen.
		
		Das Projekt selbst steht unter einer \Gls{mit-lizenz} und kann daher von
        jedem frei verwendet werden. Wenn du selbst Verbesserungsvorschläge
        hast oder dich an dem Projekt beteiligen willst, fühle dich frei, auf
        \Gls{github} ein \emph{Issue} oder direkt ein \emph{Pull Request} zu
        öffnen, sodass diese Vorlage weiterhin stetig verbessert wird. Zum 
        \Gls{github}-Repository gelangst du mit folgender \ac{url}: 
		\url{https://github.com/dateiexplorer/dhbw-latex-template}.
	\end{abstract}
\end{otherlanguage}

\begin{otherlanguage}{english}
	\begin{abstract}
		This document provides an overview of various functions of the template
        for thesises at the \acs{dhbw}. Furthermore, this document was
        created using the template for thesises and thus serves as a small
        documentation and as a reference for various \LaTeX{} commands to get a
        quick entry even without prior knowledge about \LaTeX{}.
		
		The project itself is licensed under MIT and can therefore be used
        freely by anyone. If you have any suggestions for improvement or if you
        want to contribute to the project feel free to open an \emph{Issue} or
        directly a \emph{Pull Request} on \Gls{github}, so that this template
        will continue to be constantly improved. You can get to the \Gls{github}
        repository with the following \ac{url}: 
		\url{https://github.com/dateiexplorer/dhbw-latex-template}.
	\end{abstract}
\end{otherlanguage}

%------------------------------------------------------------------------------%
% Add more abstracts...