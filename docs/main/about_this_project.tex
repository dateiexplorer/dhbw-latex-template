%!TEX root = ../../thesis.tex

\chapter{Über dieses Projekt}
    
Ziel dieses Projekts ist es, eine leichtgewichtige und moderne 
\LaTeX{}-Vorlage für wissenschaftliche Arbeiten an der \Gls{dhbw}
bereitzustellen. Dabei soll \LaTeX{} vor allem auch für Neulinge zugänglich 
gemacht werden, sodass die Hürde der Einarbeitung möglichst gering wird. 
Deshalb ist sämtlicher \LaTeX{}-Code ausführlich dokumentiert, sodass alle 
Funktionen genau erläutert werden.

Diese \LaTeX{}-Vorlage wird von Grund auf neu geschrieben und hat das Ziel, 
so wenige Pakte wie möglich einzubinden und diese so zu strukturieren, dass 
sie modular inkludiert werden können.

Dieses Dokument selbst dient nicht primär zur Dokumentation des Codes, 
sondern stellt lediglich einige Beispiele bereit, um alle benötigten 
Funktionen für das Schreiben einer wissenschaftlichen Arbeit vorzustellen.
Dabei wurde dieses Dokument selbst mit der Vorlage erstellt.
Dieses Dokument ist also vielmehr als eine Sammlung zu verstehen, in der
nachgeschlagen werden kann, wenn eine bestimmte Funktion in eigenen
Arbeiten übernommen werden soll. Jede Funktion hat hierfür ein eigenes 
Kapitel. Größere Themenkomplexe sind in Kapitel und Unterkapitel 
aufgeteilt. Dieses Dokument zeigt also, wie eine wissenschaftliche Arbeit
aussehen könnte.

\section{Disclamer}

Vorweg: Diese Vorlage ist keine offizielle Vorlage irgendeiner \Gls{dhbw}.
Deshalb erhebt sie auch keinen Anspruch auf Vollständigkeit oder
Richtigkeit. Es sollten vorher die Anforderungen an die Arbeit bei der
Hochschule geprüft werden. Ggf. müssen einige Änderungen vorgenommen werden.
Dennoch versucht diese Vorlage die Grundanforderungen der \Gls{dhbw}
umzusetzen.

\section{Entwicklung}

Dieses Projekt steht unter der \Gls{mit-lizenz}. Damit steht dieses Projekt
jedem (zur Nutzung und zur Weiterentwicklung) frei zur Verfügung. Genauere
Details zur Lizenz finden sich in einer entsprechenden separaten Datei.

Natürlich steht es jedem frei, an dieser Vorlage selbst weiterzuentwickeln.
Dabei möchte ich jedoch darauf hinweisen, dass viele Probleme eventuell auch
andere Studenten betreffen. Deshalb ist es ausdrücklich erwünscht,
Weiterentwicklungen der Vorlage oder bestimmte Features per Pull-Request
wieder in dieses Repository zurückfließen zu lassen. Diese Vorlage ist ein
Community-Projekt und lebt davon, dass es viele Entwickler gibt, die ihren
Teil dazu beitragen.