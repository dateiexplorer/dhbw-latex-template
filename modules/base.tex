%!TEX root = ../thesis.tex

%-----------------------------------------------------------------------------%
% DOCUMENT
%-----------------------------------------------------------------------------%
\documentclass[
    pdftex,
    % Print only one side of a paper, alternate: twoside
    oneside,
    % Font size (recommended by DHBW)
    12pt,
    % Don't indent space for each paragraph
    parskip=half,
    % Needed to add indices to tableofcontents
    listof=totoc,
    bibliography=totoc,
    % Change the header and footer height
    headheight=26pt,
    footheight=16pt,
    %
    headinclude=false,
    footinclude=false,
    % Add a seperation line for the header
    headsepline,
    % Some modifications for the pdf document
    % Needed the `geometry` package loaded and margin set to work correctly.
    DIV=calc,
    BCOR=8mm,
    appendixprefix
]{scrreprt}

%-----------------------------------------------------------------------------%
% ENCODING
%-----------------------------------------------------------------------------%
% Support other language encodings, e.g. 'ä', 'ö', 'ü'. Include this packages 
% in every latex document.
\usepackage[utf8]{inputenc}
\usepackage[T1]{fontenc}

%-----------------------------------------------------------------------------%
% PACKAGES
%-----------------------------------------------------------------------------%

% Use the following babel packages, set last as default.
% Add other languages if necessary.
\usepackage[english, ngerman]{babel}

% Use prettier font.
% Other fonts: palatino, goudysans, lmodern, libertine
\usepackage{lmodern}

% Change the default margin of the document. You can change the values for your
% beheviour
\usepackage[foot=1.0cm, margin=2.5cm]{geometry}

% Include figures.
\usepackage{graphicx}

% Needed to place images at code position
\usepackage{float}

% Use strings and make string operations, such as \IfStrEq available.
\usepackage{xstring}

% Use substring operations.
% Needed to know how many authors writing on the thesis.
\usepackage{substr}

% Enable `\rgb` command to define custom colors.
% Needed by some modules, e.g. for listings.
\usepackage{xcolor}
%!TEX root = ../thesis.tex

%-----------------------------------------------------------------------------%
% COLORS
%-----------------------------------------------------------------------------%

\definecolor{background}{rgb}{0.94, 0.94, 0.94}
\definecolor{codegray}{rgb}{0.5,0.5,0.5}

% Enables onehalfspacing. The default value should work for the most thesises.
% Other options are `doublespacing` and `singlespacing`. You can turn other
% space settings on by typing e.g. \doublespacing before a phrase.
% It is recommended to use this package, so it is in the base module.
\usepackage[onehalfspacing]{setspace}

% Use \cite, \parencite, for citations.
\usepackage{csquotes}

% The next three packages are useful to enable hyperrefs and bookmarking for pdf
% documents.
% Don't delete any of these packages and don't rearrange them. This could cause
% unexpected issues.
\usepackage{pdfpages}

% Commands for if statements.
\usepackage{ifthen}

% Enable the ability to use \pagestyle{fancy} with custom headers.
% This package is optional but is installed in most LaTeX distribution by
% default.
\usepackage{fancyhdr}

%-----------------------------------------------------------------------------%
% COMMANDS
%-----------------------------------------------------------------------------%
%!TEX root = ../thesis.tex

%-----------------------------------------------------------------------------%
% COMMANDS
%-----------------------------------------------------------------------------%

% Document language
\newcommand{\setDocumentLanguage}[1]{\def\documentLanguage{#1}}

% Document type
\newcommand{\setDocumentType}[1]{\def\documentType{#1}}

% Title
\newcommand{\setDocumentTitle}[1]{\def\documentTitle{#1}}

% Author
\newcommand{\setDocumentAuthor}[1]{\def\documentAuthor{#1}}

% Matriculation number
\newcommand{\setMatriculationNumber}[1]{\def\matriculationNumber{#1}}

% Course name
\newcommand{\setCourse}[1]{\def\course{#1}}

% Location of the university
\newcommand{\setLocationUniversity}[1]{\def\locationUniversity{#1}}

% Release date
\newcommand{\setReleaseDate}[1]{\def\releaseDate{#1}}

% Release location
\newcommand{\setReleaseLocation}[1]{\def\releaseLocation{#1}}

% Period
\newcommand{\setDocumentPeriod}[1]{\def\documentPeriod{#1}}

% Department
\newcommand{\setDepartment}[1]{\def\department{#1}}

% Lecture
\newcommand{\setLecture}[1]{\def\lecture{#1}}

% Degree
\newcommand{\setDegree}[1]{\def\degree{#1}}

% Name of the tutor
\newcommand{\setTutor}[1]{\def\tutor{#1}}

% Name of the evaluator
\newcommand{\setEvaluator}[1]{\def\evaluator{#1}}

% Company name
\newcommand{\setCompanyName}[1]{\def\companyName{#1}}

% Company location
\newcommand{\setCompanyLocation}[1]{\def\companyLocation{#1}}

% Restriction
\newcommand{\restrictDocument}[1]{\def\restricted{#1}}

% Electronic
\newcommand{\hasElectronicVersion}[1]{\def\electronicVersion{#1}}

% Company logo
\newcommand{\showCompanyLogo}[1]{\def\companyLogo{#1}}

% DHBW logo
\newcommand{\showDHBWLogo}[1]{\def\dhbwLogo{#1}}

% Point out code
\newcommand{\setEmphraseCode}[1]{\def\emphraseCode{#1}}

%-----------------------------------------------------------------------------%

% Check document type.
\newcommand{\IfDocType}[3]{%
    \IfStrEq{\documentType}{#1}{#2}{#3}%
}%

% Check if string is empty.
\newcommand{\IfStrIsEmpty}[3]{%
    \IfStrEq{#1}{}{#2}{#3}%
}%

% Write text in code style.
\newcommand{\code}[1]{%
    \IfStrEq{\emphraseCode}{yes}{%
        \colorbox{background}{\texttt{#1}}%
    }{% else
        \texttt{#1}%
    }
}%

%-----------------------------------------------------------------------------%

% Generate appendix table of contents
% Working solution without extra packages (KOMA) from:
% https://tex.stackexchange.com/questions/260445/separate-table-of-contents-for-appendix
\DeclareNewTOC[%
    owner=\jobname,
    listname={\appendixname}, % title of the appendix toc
]{atoc}

\makeatletter
\newcommand*\appendixwithtoc{%
    \cleardoublepage%
    \appendix%
    \addcontentsline{toc}{chapter}{\appendixname}%
    \renewcommand*{\ext@toc}{atoc}%
    \scr@ifundefinedorrelax{hypersetup}{}{\hypersetup{bookmarkstype=atoc}}%
    \listofatocs
}
\makeatother

%-----------------------------------------------------------------------------%

%-----------------------------------------------------------------------------%
% SETTINGS
%-----------------------------------------------------------------------------%
%!TEX root = ../document.tex

%-----------------------------------------------------------------------------%
% Settings marked with * are required.
%-----------------------------------------------------------------------------%

% Document language*
% If the language is not supported yet, you can define your own language file or
% use the default (en_US).
\setDocumentLanguage{de_DE}

% Document type*
% Available types:
% t1000    - Template for T3_1000 (project thesis, evaluator from company)
% t2000    - Template for T3_2000 (project thesis, evaluator from university)
% t3000    - Template for T3_3000 (project thesis, evaulator from company)
% seminar  - Template for T3_3300 (seminar thesis, evaluator from university)
% bachelor - Template for bachelor thesis (evaluator from company and
%            university)
\setDocumentType{Vorlage}

% Title*
% The title should not be longer than two lines.
\setDocumentTitle{Dokumentation der Vorlage für wissenschaftliche Arbeiten an
der Dualen Hochschule Baden-Württemberg mit Beispielen}

% Author*
\setDocumentAuthor{dateiexplorer}

% Matriculation number*
% Usually a seven-digit number, e.g. 5703134
\setMatriculationNumber{xxxxxx}

% Course name*
% Find in on your student ID, e.g. MOS-TINF19B
\setCourse{xxxxxxxxxx}

% Location of the university*
\setLocationUniversity{xxxxxxxxxxxxxxxxxxxx}

% Release date*
\setReleaseDate{\today}

% Release location*
\setReleaseLocation{xxxxxxxxxxxxxxxxxxxx}

% Period*
% The duration the thesis was written through.
\setDocumentPeriod{10. Mai 2021 - \today}

% Department*
% e.g. Angewandte Informatik
\setDepartment{xxxxxxxxxxxxxxxxxxxx}

% Lecture
% Optional, if this thesis was written for a specific lecture.
\setLecture{}

% Degree (required if \documentType{bachelor})
\setDegree{xxxxxxxxxxxxxxxxxxxx}

% Name of the tutor*
\setTutor{xxxxxxxxxxxxxxxxxxxx}

% Name of the evaluator (required if \documentType{bachelor})
\setEvaluator{}

% Company name
% Leave empty if not required. Only in combination with \setCompanyLocation.
\setCompanyName{xxxxxxxxxxxxxxxxxxxx}

% Company location
% Leave empty if not required. Only in combination with \setCompanyName.
\setCompanyLocation{xxxxxxxxxxxxxxxxxxxx}

% Restriction
% Is this document restricted to the public? Set `yes` if so.
\restrictDocument{no}

% Electronic
% Has this document an electronic version? Set `yes` if so.
\hasElectronicVersion{yes}

%-----------------------------------------------------------------------------%
% DHBW logo
% Show the DHBW logo on the titlepage? Set `yes` if so.
%
% If the DHBW logo is not rendered correctly, you can change the size inside of
% the cover module in `modules/cover.tex`
\showDHBWLogo{yes}

% Company logo
% Show the company logo on the titlepage? Set `yes` if so.
%
% If the company logo is not rendered correctly, you can change the size inside
% of the cover module in `modules/cover.tex`
\showCompanyLogo{yes}

% Emphrase code
% Set `yes` if inline \code should be emphrased with gray background.
\setEmphraseCode{yes}

% Quote style
% Set the style of the quotes.
%
% Available types:
% harvard - e.g. (autor year, p.10)
% default - e.g. [1, p.10]
%
% If no style is set, the template will use the default template.
\setQuoteStyle{default}

%-----------------------------------------------------------------------------%
% LANGUAGE
%-----------------------------------------------------------------------------%
% Load the language file, if not exists, load default.
\edef\BaseFile{res/i18n/\documentLanguage/base}%
\input{\BaseFile}

\usepackage[
    pdftitle={\documentTitle},
    pdfauthor={\documentAuthor},
    pdfsubject={\documentType},
    pdfcreator={pdflatex, LaTeX with KOMA-Script},
    pdfpagemode=UseOutlines,
    pdfdisplaydoctitle=true,
    pdflang={\languageShortId},
    hidelinks
]{hyperref}

\usepackage{bookmark}

%-----------------------------------------------------------------------------%
% MODULES
%-----------------------------------------------------------------------------%
% The modules are loaded exactly here. This is necessary because some packages
% need other packages and some must be load bevor load others (look at the
% fixes) to work correctly.
% The default beheviour is to load all recommended packages. If you don't want
% or need some packages, comment it out in the modules file shown below.
%!TEX root = ../thesis.tex

%-----------------------------------------------------------------------------%
% All available modules are listed below. To disable a module, just comment it
% out with a `%` sign at the beginning of the line.
% If you want to know, which code needs this package to be load, search for the
% `#<string>`.
%-----------------------------------------------------------------------------%

%!TEX root = ../thesis.tex

%------------------------------------------------------------------------------%
% ABSTRACT
%------------------------------------------------------------------------------%

\usepackage{abstract}

% Define your abstract in a separate file.
% This module must be added before the `\begin{document}` tag. % #abstract
%!TEX root = ../document

\usepackage[backend=biber]{biblatex}

% Add your bibliography files here...
\addbibresource{bibliography.bib} % #bibliography
%!TEX root = ../thesis.tex

%-----------------------------------------------------------------------------%
% GLOSSARIES
%-----------------------------------------------------------------------------%
% Make glossaries
\usepackage[nonumberlist, toc]{glossaries}

\makeglossaries
 % #glossaries
%!TEX root = ../thesis.tex

%------------------------------------------------------------------------------%
% ACRONYMS
%------------------------------------------------------------------------------%
% Make acronyms
\usepackage[printonlyused]{acronym}
 % #acronyms
%!TEX root = ../thesis.tex

%------------------------------------------------------------------------------%
% LISTINGS
%------------------------------------------------------------------------------%
\usepackage{listings}

\lstdefinestyle{defaultstyle}{
    backgroundcolor=\color{background},
    numberstyle=\tiny\color{codegray},
    basicstyle=\footnotesize\ttfamily,
    breakatwhitespace=false,
    breaklines=true,
    captionpos=t,
    keepspaces=true,
    numbers=left,
    numbersep=5pt,
    showspaces=false,
    showstringspaces=false,
    showtabs=false,
    tabsize=4
}

\lstset{style=defaultstyle} % #listings
%!TEX root = ../thesis.tex

%-----------------------------------------------------------------------------%
% APPENDIX
%-----------------------------------------------------------------------------%
% Define some util packages for appendix

% Availability to include pdf documents in the appendix
\usepackage{pdfpages}

\includepdfset{pagecommand={\thispagestyle{headings}}} % #appendix
%!TEX root = ../thesis.tex

%-----------------------------------------------------------------------------%
% MATHEMATICS
%-----------------------------------------------------------------------------%
%
% This module provides some packages for mathematical stuff like equations,
% matrices, etc.

\usepackage{amsmath}
\usepackage{amssymb} % #mathematics

%-----------------------------------------------------------------------------%
% FIXES
%-----------------------------------------------------------------------------%
% By default this template enables all fixes to improve the compiled result.
% However, if you have any problems or don't want the beheviour, just comment
% out the fix with a `%` sign.
% It is recommended to enable all fixes for the best results.
%!TEX root = ../../thesis.tex

%------------------------------------------------------------------------------%
% This fix improves the typography and layout of the pdf file.
%------------------------------------------------------------------------------%

% Fix: Pin chapter headings at top of page.
\renewcommand*\chapterheadstartvskip{}

% Avoid `Schusterjungen` and `Hurenkinder`
\clubpenalty=10000
\widowpenalty=10000
%!TEX root = ../../thesis.tex

%-----------------------------------------------------------------------------%
% This fix adds support to sans serif fonts.
%-----------------------------------------------------------------------------%

% Load fonts
\usepackage{sourcesanspro}
\usepackage{sourcecodepro}

% Apply sans serif font as default
\renewcommand*{\familydefault}{\sfdefault}

% Set URL to sans serif font
\urlstyle{sf}
%!TEX root = ../../thesis.tex

%-----------------------------------------------------------------------------%
% This fix add a package `scrhack` to supress the deprecation warnings from
% inside the KOMA-scripts.
%-----------------------------------------------------------------------------%

\usepackage{scrhack}
%!TEX root = ../../thesis.tex

%------------------------------------------------------------------------------%
% This fix uses the microtype package to enable better typography and decrease
% `Overfull` and `Underfull` information output from latex.
%
% If using a bibliography with the biblatex package, the url handling of xurl
% is borken. To solve this issue, load the xurl package after the biblatex
% package.
% https://tex.stackexchange.com/questions/3033/forcing-linebreaks-in-url
%------------------------------------------------------------------------------%

% Enable better typograhpie to decrease `Overfull` and `Underfull` informations.
\usepackage[activate]{microtype}

% Load this package after `biblatex` to solve url linebreaking in the
% bibliography.
\usepackage{xurl}
%!TEX root = ../../thesis.tex

%------------------------------------------------------------------------------%
% This fix provides internationalization for listings.
%------------------------------------------------------------------------------%

\renewcommand{\lstlistingname}{\sListingPhrase}
\renewcommand{\lstlistlistingname}{\sListListingPhrase}

% Do not comment out this fixes, otherwise some other pieces would not work.
%!TEX root = ../../thesis.tex

%------------------------------------------------------------------------------%
% Support multiple authors for thesis.
% Separate authors by a colon.
%------------------------------------------------------------------------------%

% Check if multiple authors are included.
\newboolean{multipleAuthors}%
\IfCharInString{,}{\documentAuthor}{%
    \setboolean{multipleAuthors}{true}%
}{%
    \setboolean{multipleAuthors}{false}%
}%
%!TEX root = ../../thesis.tex

%------------------------------------------------------------------------------%
% Add a new command \source to add sources to figures.
%------------------------------------------------------------------------------%

% Add source to figures.
\newcommand{\source}[1]{%
    \footnotesize{\sSourcePhrase{}: #1}%
}%